\documentclass{article}
\usepackage[utf8]{inputenc}
\usepackage{fullpage}
\usepackage{hyperref}
\hypersetup{
    colorlinks=true,
    linkcolor=blue,
    filecolor=magenta,      
    urlcolor=cyan,
    pdftitle={Overleaf Example},
    pdfpagemode=FullScreen,
    }

\title{De-mystifying the World Economic Forum (WEF)}
\author{Tyler Reed }
\date{Due: February 25, 2022}


\begin{document}

\maketitle{}

\section*{Motivation}
Understanding of the membership structure, agendas of the World Economic Forum (WEF) and their impact on Western society's democratic ideals is obscure among the general population. In 2020, WEF had 627 self-selected, not elected, member corporations. Over 3,000 invitation-only participants attended the flagship annual Davos Meeting in Switzerland, including 410 public figures. Since WWII, the United Nations (U.N.) has served as the foremost international body for geopolitical and economic issues for many countries, primarily those with Western democratic ideals. WEF members could pose serious concerns to members' democratic nations when many such nations already have economic representation through the democratic process within the United Nations (U.N.). The main cause of concern is the makeup of the members of WEF compared to those of the United Nations. World citizens, especially Westerners, may be interested in knowing who WEF is by the current makeup of corporate members by national representation. Providing a visualization of the current makeup of WEF could inform citizens of those types of corporate entities that join WEF and those that do not, including industry sector. Then citizens will be able to further their investigation of WEF, any members of interest, and the scope of impact WEF has in their individual lives. The tasks at hand will attempt to provide means whereby world citizens can be informed of the membership structure, agenda of complex, unfamiliar organizations, such as WEF, via visualizations.
% more details about what precisely the concern is...
\\

% * Know geographic rep of membership
%     * What nations are they from?
%     * Do they overrepresent dev nations and underrepresent undeveloped?
% * Representation compared to UN
%     * Are all nations represented in WEF, where’s the gap?
% * Understand history of agendas

\subsection*{Task 1}
\underline{International Representation of Member Entities}
\\

Knowing the current international representation of membership and the representation of corporate entities vs government or other entities will aid in determining the scope of influence.
% specify that it's just for WEF
% rephrase as how many corporations vs government vs other entities countries tend to have in the WEF (make wording more specific to a distribution)
\\
\\
\href{https://www.weforum.org/partners}{WEF Partner/Member Entities}

\subsection*{Task 2}
\underline{Comparing the Membership of the UN to WEF}
\\

Comparing the current difference in representation across different nations for UN and WEF membership
% will better inform citizens of the representation difference.
% distribution implies that you're looking at some spread of data, whereas representation is a fixed number...
\\
\\
\href{https://www.weforum.org/about/history}{WEF History}

\subsection*{Task 3}
\underline{WEF Mission}
\\

Comprehend a general sense of how the agendas of WEF have changed over time since its inception in 1971. Annual reports and timeline data will be sourced.
\\
\\
\href{https://www.weforum.org/about/world-economic-forum}{WEF Mission}

\section*{Literature Review}

\begin{itemize}
\item Task 1
\item Task 2
\item Task 3
\end{itemize}




\section*{Data Source}

Online, pre-processed datasets of the membership, structure, or mission  of WEF are not readily available. Scraping the WEF website seems to be the most appropriate method to source the necessary data. 



\end{document}